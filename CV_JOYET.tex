\documentclass[11pt,a4paper,sans]{moderncv}
\moderncvstyle{classic} 
\moderncvcolor{green}
\setlength{\hintscolumnwidth}{2.5cm} 
\usepackage[utf8]{inputenc}
\usepackage[T1]{fontenc} 
\usepackage[scale=0.9]{geometry}
\usepackage{helvet}
\usepackage[french]{babel}

\newlength\Tripleitemmaincolumnwidth
\newlength\tripleitemmaincolumnwidth

\AtBeginDocument{%
\setlength\tripleitemmaincolumnwidth{% 
  \maincolumnwidth-2\hintscolumnwidth-2\separatorcolumnwidth}%
\setlength\tripleitemmaincolumnwidth{.333333\tripleitemmaincolumnwidth}%   
  }

\newcommand*{\cvtripleitem}[7][.25em]{%
 \cvitem[#1]{#2}{%
   \begin{minipage}[t]{\tripleitemmaincolumnwidth}#3\end{minipage}%
   \hfill
   \begin{minipage}[t]{\hintscolumnwidth}\raggedleft\hintstyle{#4}\end{minipage}%
   \hspace*{\separatorcolumnwidth}%
   \begin{minipage}[t]{\tripleitemmaincolumnwidth}#5\end{minipage}%
   \hfill
   \begin{minipage}[t]{\hintscolumnwidth}\raggedleft\hintstyle{#6}\end{minipage}%
   \hspace*{\separatorcolumnwidth}%
   \begin{minipage}[t]{\tripleitemmaincolumnwidth}#7\end{minipage}%
   }%
}

%%% Info Persos
\name{\huge{Pierre}}{\huge{JOYET}}
\extrainfo{16 mai 1996}
\email{pierre.joyet@pm.me}
\homepage{pjoyet.github.io/}
\social[linkedin]{www.linkedin.com/in/pierre-joyet}
\photo[90pt][0.7pt]{pierre.jpg}

%%% Titre
\begin{document}
\makecvtitle 

%%% EXPERIENCE PROFESSIONELLE
\section{Expériences professionnelles}

\cventry{\textbf{2022} mars-août}{CDD INGÉNIEUR D'ÉTUDES}{Institut national de recherche pour l'agriculture, l'alimentation et l'environnement (INRAE)}{Unité Technologies et systèmes d'information pour les agrosystèmes (TSCF)}{Centre Clermont Auvergne-Rhône-Alpes}{}{}
\vspace{0.2cm}

\cventry{\textbf{2021} mars-août}{STAGE}{INRAE}{Unité TSCF}{Centre Clermont Auvergne-Rhône-Alpes}{Etude de commande en coopération des manipulateurs mobiles à roues.}{}
\vspace{0.2cm}

\cventry{\textbf{2019} janvier-mai}{STAGE}{Institut National de la Recherche Agronomique}{Laboratoire de Génie et Microbiologie des Procédés Alimentaires}{Campus AgroParisTech de Thiverval-Grignon}{Mise au point d'un appareillage de suivi d'échantillons par spectrométrie dans le proche infrarouge et le visible, gestion d'un cahier des charges.}
\vspace{0.2cm}

%%% FORMATION
\section{Formation}

\cventry{\textbf{2019-2021}}{MASTER Automatique, Robotique, parcours Systèmes Avancés et Robotique}{Sorbonne~Université (ex UPMC Paris 6), Faculté des Sciences et Ingénierie, en collaboration avec Arts~et~Métiers ParisTech (ENSAM) et MINES ParisTech}{Paris}{}{}
\vspace{0.2cm}

\cventry{\textbf{2014-2019}}{LICENCE de Sciences, Technologies, Santé, mention Mécanique}{Département d'Ingénierie Mécanique de Sorbonne Université}{Paris}{}{}

%%% Langues
\section{Langues}

\cvdoubleitem{\textbf{Français}}{Langue maternelle}{\textbf{Anglais}}{Niveaux B2}

% Informatique
\section{Compétences}

\cvdoubleitem{\textbf{Logiciels}}{Solidworks\\FreeCAD\\Suite Office\\Git}{\textbf{Langages}}{C++\\ Python/orienté objet\\ \LaTeX\\ Matlab / Octave\\ Fortran 90}
\vspace{0.2cm}

\cvdoubleitem{\textbf{Systèmes}}{ROS\\Linux}{\textbf{Supports}}{Universal Robots\\Imprimante 3D\\ Arduino\\ Raspberry pi}{}{}

% Loisirs
\section{Centres d'intérêts}

\cvdoubleitem{\textbf{Artistique}}{Dessin (9 ans), Peinture (9 ans),\\ Vitrail (6 ans)}{\textbf{Musique}}{Piano (5 ans)}

\cvitem{\textbf{Patrimoine}}{Chantier de restauration - Union Rempart (6 sessions)}
\end{document}
