\documentclass[11pt,a4paper,sans]{moderncv}
\moderncvstyle{classic} 
\moderncvcolor{green}
\setlength{\hintscolumnwidth}{2.5cm} 
\usepackage[utf8]{inputenc}
\usepackage[T1]{fontenc} 
\usepackage[scale=0.9]{geometry}
\usepackage{helvet}
\usepackage[french]{babel}

\newlength\Tripleitemmaincolumnwidth
\newlength\tripleitemmaincolumnwidth

\AtBeginDocument{%
\setlength\tripleitemmaincolumnwidth{% 
  \maincolumnwidth-2\hintscolumnwidth-2\separatorcolumnwidth}%
\setlength\tripleitemmaincolumnwidth{.333333\tripleitemmaincolumnwidth}%   
  }

\newcommand*{\cvtripleitem}[7][.25em]{%
 \cvitem[#1]{#2}{%
   \begin{minipage}[t]{\tripleitemmaincolumnwidth}#3\end{minipage}%
   \hfill
   \begin{minipage}[t]{\hintscolumnwidth}\raggedleft\hintstyle{#4}\end{minipage}%
   \hspace*{\separatorcolumnwidth}%
   \begin{minipage}[t]{\tripleitemmaincolumnwidth}#5\end{minipage}%
   \hfill
   \begin{minipage}[t]{\hintscolumnwidth}\raggedleft\hintstyle{#6}\end{minipage}%
   \hspace*{\separatorcolumnwidth}%
   \begin{minipage}[t]{\tripleitemmaincolumnwidth}#7\end{minipage}%
   }%
}
\newcounter{age} \setcounter{age}{\the\year} \addtocounter{age}{-1996}
%%% Info Persos
%\title{\large{Candidature pour la thèse :\\"Contrôle optimal d'outils robotiques pour la réalisation de travaux agroécologiques"}}
\name{\huge{Pierre}}{\huge{JOYET}} %Roboticien\\\vspace{0.2cm} 
\extrainfo{16 mai 1996}% (\arabic{age} ans)

% \extrainfo{16 mai 1996\\Permis B en cours d'acquisition}% (\arabic{age} ans)
% \address{59 rue Caulaincourt}{75018, Paris}{France}
%\address{36 rue Neyron}{63000, Clermont-Ferrand}{France}
%\phone[mobile]{+336 09 95 77 86}
\email{pierre.joyet@pm.me}
\homepage{pjoyet.github.io/}
\social[linkedin]{www.linkedin.com/in/pierre-joyet}
\photo[90pt][0.7pt]{pierre3.jpg}

%%% Titre

\begin{document}
\makecvtitle 


%%% EXPERIENCE PROFESSIONELLE
\section{Expériences professionnelles}% et techniques

\cventry{\textbf{2022} mars-août}{CDD INGÉNIEUR D'ÉTUDES}{Institut national de recherche pour l'agriculture, l'alimentation et l'environnement (INRAE)}{Unité Technologies et systèmes d'information pour les agrosystèmes (TSCF)}{Centre Clermont Auvergne-Rhône-Alpes}{}{}%\newline{}Compétences acquises: utilisation de robot industriel
\vspace{0.2cm}

\cventry{\textbf{2021} mars-août}{STAGE}{INRAE}{Unité TSCF}{Centre Clermont Auvergne-Rhône-Alpes}{Etude de commande en coopération des manipulateurs mobiles à roues.}{}%\newline{}Compétences acquises: utilisation de robot industriel
\vspace{0.2cm}

% \cventry{\textbf{2021} janvier-février}{PROJET de fin d'étude}{Sorbonne Université}{Paris}{}{Conception d'une commande de collaboration entre une plateforme mobile terrestre et un drone.}{}%\newline{}Compétences acquises: 
%\vspace{0.2cm}


% \cventry{\textbf{2020} juin-juillet}{PROJET}{Sorbonne Université}{Paris}{}{ Etat de l’art des cinématiques des rovers et analyse de leur capacité de franchissement.}{}
%\vspace{0.2cm}

\cventry{\textbf{2019} janvier-mai}{STAGE}{Institut National de la Recherche Agronomique}{Laboratoire de Génie et Microbiologie des Procédés Alimentaires}{Campus AgroParisTech de Thiverval-Grignon}{Mise au point d'un appareillage de suivi d'échantillons par spectrométrie dans le proche infrarouge et le visible, gestion d'un cahier des charges.}%\newline{}Compétences acquises: language Python, Raspberry pi, impression 3D.
\vspace{0.2cm}

%\cventry{\textbf{2016} janvier-juin\\1 jour/sem}{PROJET RoMarin}{Sorbonne Université}{Paris}{}{Projet en équipe, au sein d'une unité d'enseignement, de construction et d'optimisation d'un robot sous-marin (ROV), spécialisation dans la construction d'une pince robotique.}{}%\newline{}Compétences acquises: utilisation de Solidworks, Arduino.
%\vspace{0.2cm}

%\cventry{\textbf{2016-2019} 1-3 sem/an}{BENEVOLAT}{Union Rempart}{Berzy-le-Sec}{Aisne}{Restauration du Patrimoine, apprentissage du travail en groupe.}


%%% ÉDUCATION

\section{Formation}
\cventry{\textbf{2019-2021}}{MASTER Automatique, Robotique, parcours Systèmes Avancés et Robotique}{Sorbonne~Université (ex UPMC Paris 6), Faculté des Sciences et Ingénierie, en collaboration avec Arts~et~Métiers ParisTech (ENSAM) et MINES ParisTech}{Paris}{}{}
\vspace{0.2cm}
\cventry{\textbf{2014-2019}}{LICENCE de Sciences, Technologies, Santé, mention Mécanique}{Département d'Ingénierie Mécanique de Sorbonne Université}{Paris}{}{}
%\cventry{année--année}{Titre de l'emploi}{Employeur}{Ville}{}{Description de 1 ou 2 lignes.\newline{}}%


%%% Langues
\section{Langues}
\cvdoubleitem{\textbf{Français}}{Langue maternelle}{\textbf{Anglais}}{Niveaux B2}

% Informatique
\section{Compétences}
%\cvdoubleitem{\textbf{Langages}}{C++\\ Python/orienté objet\\ Matlab\\ Fortran 90\\ LaTeX}{\textbf{Ingénierie}}{Méthodologie Numériques\\ Ingénierie Robotique\\ Intelligence Artificielle}

%\cvdoubleitem{\textbf{Logiciels}}{Solidworks\\ROS\\FreeCAD\\Suite Office}{\textbf{Support}}{Imprimante 3D\\ Arduino\\ Raspberry pi}{}{}
%\vspace{0.2cm}
%\cvdoubleitem{\textbf{Langages}}{C++\\ Python/orienté objet\\ Matlab\\ Fortran 90\\ LaTeX}{}{}

%\cvtripleitem{\textbf{Logiciels}}{Solidworks\\ROS\\FreeCAD\\Suite Office}{\textbf{Langages}}{C++\\ Python/orienté objet\\ Matlab\\\LaTeX \\Fortran 90}{\textbf{Support}}{Imprimante 3D\\ Arduino\\ Raspberry pi}

\cvdoubleitem{\textbf{Logiciels}}{Solidworks\\FreeCAD\\Suite Office\\Git}{\textbf{Langages}}{C++\\ Python/orienté objet\\ \LaTeX\\ Matlab / Octave\\ Fortran 90}
\vspace{0.2cm}
\cvdoubleitem{\textbf{Systèmes}}{ROS\\Linux}{\textbf{Supports}}{Universal Robots\\Imprimante 3D\\ Arduino\\ Raspberry pi}{}{}



% Loisirs
\section{Centres d'intérêts}
\cvdoubleitem{\textbf{Artistique}}{Dessin (9 ans), Peinture (9 ans),\\ Vitrail (6 ans)}{\textbf{Musique}}{Piano (5 ans)}
\cvitem{\textbf{Patrimoine}}{Chantier de restauration - Union Rempart (6 sessions)}
%\cvitem{\textbf{Lecture}}{Science fiction, space opéra, heroic fantasy}{das}{sad}
\end{document}
